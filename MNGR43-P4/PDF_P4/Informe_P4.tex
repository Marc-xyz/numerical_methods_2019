\documentclass[paper=a4, spanish, fontsize=12pt]{scrartcl}
 
%Paquets d'idioma i codificaci� de car�cters
\usepackage[T1]{fontenc}
\usepackage[spanish,es-tabla]{babel}

%Paquets d'escriptura matem�tica
\usepackage{amsmath,amsfonts,amssymb,amsthm}

%Modificacions i noves comandes
\renewcommand\qedsymbol{QED} %Canviem el s�mbol de demostraci� final  de un quadrat en blanc a QED 
\newcommand{\C}{\ensuremath{\mathbb{C}}}
\newcommand{\R}{\ensuremath{\mathbb{R}}}
\newcommand{\Q}{\ensuremath{\mathbb{Q}}}
\newcommand{\Z}{\ensuremath{\mathbb{Z}}}
\newcommand{\N}{\ensuremath{\mathbb{N}}}
\newcommand{\B}{\ensuremath{\mathcal{B}}}
\newcommand{\esp}{\text{ }}
\newcommand{\cod}[1]{{ \color{redviolet}\texttt{#1}}}
\newcommand{\prov}[1]{{ \color{olive}\texttt{#1}}}

%Paquets de plantilla per pr�ctiques
\usepackage{fourier} 
\usepackage{paralist}
\usepackage{eurosym}
\usepackage{graphicx}
\usepackage{caption}
\usepackage{titling}
\usepackage{sectsty} % Allows customizing section commands
\usepackage{fancyhdr} % Custom headers and footers
\usepackage{wrapfig}
%Altres paquets

\usepackage[svgnames]{xcolor}
\usepackage{multirow} %Serveix per agrupar files i columnes
\usepackage{graphicx} %Serveix per introduir imatges
\usepackage{url} % the same
\usepackage{hyperref} %Serveix per que totes les refer�ncies que apareguin en el document pdf clicant-hi amb el ratol�, el visor pdf saltar� a la posici� referenciada
\usepackage{enumerate,paralist} %Serveix per ampliar les possibilitats dels entorns de llistes
\usepackage{subfigure} %Serveix per poder generat subfigures.
\usepackage{pstricks-add} %Paquets per a dibuixos amb GeoGebra
\usepackage{centernot} %Serveix taxar coses b� com per exemple $\longrightarrow$ i $\exists$
\usepackage{colortbl} %Serveix per ficar colors a les taules
\usepackage{fancyhdr} %Serveix per fer el estil de capselera
\usepackage{stackrel} %Serveix per fotre coses a dalt i a baix
\usepackage{tcolorbox}% Serveix per ficar caixes 

%Altres dades del document
\pagestyle{fancy}
\definecolor{mygray}{rgb}{0.9,0.9,0.9} % Color definition
\definecolor{redviolet}{RGB}{5,47,75} 
\definecolor{leet}{RGB}{4,4,75}
\spanishdecimal{.}
\usepackage[all]{xy}



% Autores y su correspondiente correo
\author{Pr�ctica 4 (Cuadratura Gaussiana} % Your names
\date{\normalsize} 
\rhead{ \small \theauthor}

\setcounter{section}{-1}
\begin{document}
\begin{tcolorbox}
	{\large\textbf{Pr�ctica 4}}, {\textbf{Cuadratura Gaussiana} \small \textbf{(Trabajo final)}}  \hfill Fecha de Reporte: \textbf{27/05/2019}\\
	\textbf{M�todos num�ricos}  \hfill Alumno: \emph{Graells Ricardo, Marc}\\
	2n curso del grado en Matem�ticas
	\hfill (\textbf{\texttt{NIU:} }\textit{1388471})
\end{tcolorbox}	
\section{Resumen}
En el presente documento se comentan las soluciones al ejercicio final propuestos en las sesiones de \textit{Seminarios de M�todos Num�ricos} de los d�as 6, 13  y 20 de Mayo. Los ficheros \texttt{.c} con las implementaciones del c�digo en lenguaje C se adjuntan tal como se detalla en documento del mismo subdirectorio \texttt{README.txt}.\footnote{En el apartado, no numerado, Anexo se responden a algunos de los ejercicios complementarios.}
\footnotesize
%\tableofcontents
%\setcounter{section}{+3}
\normalsize
\section{C�lculos previos}
Consideremos las siguientes manipulaciones algebraicas elementales y definiciones de las integrales dadas. Para la primera integral:
 $$
 	F(x) \bigl\lvert_{-1}^{\esp\esp1} \esp:=\displaystyle\int_{-1}^{1}e^{-x^2}dx 
 	\qquad  \qquad \qquad \qquad  \qquad \qquad  \qquad  \qquad \qquad  \qquad \qquad  \qquad  \qquad \qquad \qquad
 	$$
 	$$
 	\qquad \qquad \qquad =\begin{cases}
 	\quad \displaystyle \int_{-1}^{1}e^{-x^2} \cdot 1 dx  \quad   \quad \quad \quad \quad \quad \quad \quad\quad \quad\Rightarrow \begin{pmatrix}f_1(x):=e^{-x^2} \\ \cod{ (C.Gauss-Legendre)}\end{pmatrix}
 	\\
 	\quad \displaystyle \int_{-1}^{1}\left(e^{-x^2} \cdot  \sqrt{1-x^2}\right) \frac{1}{\sqrt{1-x^2}}dx \quad \quad \Rightarrow \begin{pmatrix} f_2(x):=e^{-x^2}
 	 \cdot  \sqrt{1-x^2} \\ \cod{ (C.Gauss-Chebyshev)}
 	\end{pmatrix} \end{cases}
 $$
 Y para la segunda:
$$
G(x)\bigl\lvert_{\esp0}^{\esp1} \esp:=\displaystyle\int_{0}^{1}\frac{e^{-x^2}}{\sqrt[3]{1-x^2}}dx \overset{g(x)=g(-x)}{=} \frac{1}{2} \cdot \displaystyle\int_{-1}^{1}\frac{e^{-x^2}}{\sqrt[3]{1-x^2}}dx 
\qquad  \qquad \qquad \qquad  \qquad \qquad  \qquad  \qquad \qquad 
$$
$$
 \qquad \qquad  \qquad \qquad 
 = \begin{cases} \quad \displaystyle\int_{-1}^{1}\frac{1}{2}\cdot \frac{e^{-x^2}}{\sqrt[3]{1-x^2}} \cdot 1 \cdot dx  \esp \quad \quad \quad \quad \quad\quad \quad\Rightarrow
 \begin{pmatrix} g_1(x):=\frac{e^{-x^2}}{\sqrt[3]{1-x^2}} 
 \\ \cod{ (C.Gauss-Legendre)} \end{pmatrix}
\\
\quad \displaystyle \int_{-1}^{1}\ \frac{1}{2} \left(e^{-x^2} \cdot  \sqrt[6]{1-x^2}\right) \frac{1}{\sqrt{1-x^2}}dx \quad \quad \Rightarrow \begin{pmatrix} g_2(x):=e^{-x^2} \cdot  \sqrt[6]{1-x^2} \\\cod{ (C.Gauss-Chebyshev)} \end{pmatrix}
\end{cases}
$$
 \newpage
\section{Procedimiento general}
 
 Tal como se detalla en el primer punto del \textit{procedimiento general}, en este caso mediante una adaptaci�n del \textit{m�todo de la bisecci�n}, se hallan todos los intervalos en los que hay una �nica ra�z del polinomio ortogonal dado. Esto lo podemos hacer ya que por \textbf{Lema 1} sabemos que  todas las ra�ces de estos polinomios son reales y simples, por tanto esperamos $n$ intervalos si estos tienen una longitud \textit{suficientemente} peque�a.
 
 Para lograr esto, con una filosof�a heur�stica, dividimos el intervalo $[-1,1]$ en \cod{TOL} intervalos de longitud $\frac{2}{\cod{TOL}}$ y comprobamos si hay cambio de signo en cada uno de ellos (cosa que implicar�a la presencia de al menos una ra�z, por Teorema de Bolzano). 
 
 Para asegurar que en cada intervalo solo cojeemos una �nica ra�z \footnote{Equivalente a  que la longitud del intervalo sea  \textit{suficientemente} peque�a} se utilizan dos contadores, uno para el n�mero de intervalos con cambio de signo y otro para el n�mero de ra�ces halladas \footnote{Ya que estas podr�an coincidir en uno de los extremos de uno de los intervalos.}
 
 El valor \cod{TOL} puede modificar-se, por ejemplo \cod{\#define TOL 10000}  en vez de \cod{\#define TOL 1000}, aunque  para polinomios de grado $\leq 20$ no es necesario. En caso de que en alguno de los intervalos hubiera m�s de una ra�z, mediante la presencia de los contadores anteriormente mencionados, el programa imprimir�a  un mensaje de error que solventar�amos modificando el paramento \cod{TOL}.  
 
 Una vez encontrados estos intervalos eligiendo el punto medio y aplicamos el proceso iterativo de Newton-Raphson de convergencia cuadr�tica para ra�ces simples conseguiriamos las ra�ces con  tolerancia deseada o superior, sin necesidad de aplicar Sturm y asegurando que hemos encontrado todas las ra�ces del polinomio con la tolerancia deseada.
 
El sistema lineal que han de cumplir los coeficientes $a_i$ puede plantearse 
 \section{Apartado a)}
\begin{table}[ht]
	\footnotesize
	\centering
 \begin{tabular}{|c|c|c|}
 	\hline
 	Resultados para $ \begin{matrix} \\ \displaystyle\int_{-1}^{1}e^{-x^2}dx\\
 	\\
 	\end{matrix}$&\cod{ (C.Gauss-Legendre)}&\cod{ (C.Gauss-Chebyshev)}\\
 	\hline
 	\cod{n=2}&{\textbf{1.4}33062621147579}&\textbf{1}.347372359753599\\
 	\hline
 	\cod{n=4}&{\textbf{1.493}334622449539}&\textbf{1}.509544014901066\\
 	\hline
 	\cod{n=6}&{\textbf{1.49364}7614150605}&\textbf{1}.501716307252864\\
 	\hline
 	\cod{n=8}&{\textbf{1.49364826}4899014}&\textbf{1.49}8269900093957\\
 	\hline
 	\cod{n=10}&{\textbf{1.493648265624}351}&\textbf{1.49}6630466539133\\
 	\hline
 	\cod{n=12}&{\textbf{1.493648265624854}}&\textbf{1.49}5728452515196\\
 	\hline
 \end{tabular}
	\caption{\footnotesize Resultados obtenidos  mediante \cod{pr3\_abc\_d.c}}
\label{tab:pr3}
\normalsize
\end{table}
\begin{table}[ht]
	\footnotesize
	\centering
	\begin{tabular}{|c|c|c|}
		\hline
		Resultados para $ \begin{matrix} \\ \displaystyle\int_{0}^{1}\frac{e^{-x^2}}{\sqrt[3]{1-x^2}}dx\\
		\\
		\end{matrix}$&\cod{ (C.Gauss-Legendre)}&\cod{ (C.Gauss-Chebyshev)}\\
		\hline
		\cod{n=2}&\textbf{0.8}202235964492217&\textbf{0.8}487913990500435\\
		\hline
		\cod{n=4}&\textbf{0.8}653882550375894&\textbf{0}.9035877798072207\\
		\hline
		\cod{n=6}&\textbf{0.8}745199676806141&\textbf{0.8}982364172452582\\
		\hline
		\cod{n=8}&\textbf{0.8}789653183338564&\textbf{0.8}955429715534456\\
		\hline
		\cod{n=10}& \textbf{0.88}1540069130857&\textbf{0.8}940299048220692\\
		\hline
		\cod{n=12}&\textbf{0.88}31932589792649&\textbf{0.8}930783042638903\\
		\hline
	\end{tabular}
	\caption{\footnotesize Resultados obtenidos  mediante \cod{pr3\_abc\_d.c}}
	\label{tab:pr3}
	\normalsize
\end{table}
 \section{Ejercicios opcionales}
\end{document}
