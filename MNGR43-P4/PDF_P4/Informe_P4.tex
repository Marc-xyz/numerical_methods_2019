\documentclass[paper=a4, spanish, fontsize=12pt]{scrartcl}
 
%Paquets d'idioma i codificaci� de car�cters
\usepackage[T1]{fontenc}
\usepackage[spanish,es-tabla]{babel}

%Paquets d'escriptura matem�tica
\usepackage{amsmath,amsfonts,amssymb,amsthm}

%Modificacions i noves comandes
\renewcommand\qedsymbol{QED} %Canviem el s�mbol de demostraci� final  de un quadrat en blanc a QED 
\newcommand{\C}{\ensuremath{\mathbb{C}}}
\newcommand{\R}{\ensuremath{\mathbb{R}}}
\newcommand{\Q}{\ensuremath{\mathbb{Q}}}
\newcommand{\Z}{\ensuremath{\mathbb{Z}}}
\newcommand{\N}{\ensuremath{\mathbb{N}}}
\newcommand{\B}{\ensuremath{\mathcal{B}}}
\newcommand{\esp}{\text{ }}
\newcommand{\cod}[1]{{ \color{redviolet}\texttt{#1}}}


%Paquets de plantilla per pr�ctiques
\usepackage{fourier} 
\usepackage{paralist}
\usepackage{eurosym}
\usepackage{graphicx}
\usepackage{caption}
\usepackage{titling}
\usepackage{sectsty} % Allows customizing section commands
\usepackage{fancyhdr} % Custom headers and footers
\usepackage{wrapfig}
%Altres paquets

\usepackage[svgnames]{xcolor}
\usepackage{multirow} %Serveix per agrupar files i columnes
\usepackage{graphicx} %Serveix per introduir imatges
\usepackage{url} % the same
\usepackage{hyperref} %Serveix per que totes les refer�ncies que apareguin en el document pdf clicant-hi amb el ratol�, el visor pdf saltar� a la posici� referenciada
\usepackage{enumerate,paralist} %Serveix per ampliar les possibilitats dels entorns de llistes
\usepackage{subfigure} %Serveix per poder generat subfigures.
\usepackage{pstricks-add} %Paquets per a dibuixos amb GeoGebra
\usepackage{centernot} %Serveix taxar coses b� com per exemple $\longrightarrow$ i $\exists$
\usepackage{colortbl} %Serveix per ficar colors a les taules
\usepackage{fancyhdr} %Serveix per fer el estil de capselera
\usepackage{stackrel} %Serveix per fotre coses a dalt i a baix
\usepackage{tcolorbox}% Serveix per ficar caixes 

%Altres dades del document
\pagestyle{fancy}
\definecolor{mygray}{rgb}{0.9,0.9,0.9} % Color definition
\definecolor{redviolet}{RGB}{5,47,75} 
\definecolor{leet}{RGB}{4,4,75}
\spanishdecimal{.}
\usepackage[all]{xy}



% Autores y su correspondiente correo
\author{Pr�ctica 4 (Cuadratura Gaussiana} % Your names
\date{\normalsize} 
\rhead{ \small \theauthor}

\setcounter{section}{-1}
\begin{document}
\begin{tcolorbox}
	{\large\textbf{Pr�ctica 4}}, {\textbf{Cuadratura Gaussiana} \small \textbf{(Trabajo final)}}  \hfill Fecha de Reporte: \textbf{27/05/2019}\\
	\textbf{M�todos num�ricos}  \hfill Alumno: \emph{Graells Ricardo, Marc}\\
	2n curso del grado en Matem�ticas
	\hfill (\textbf{\texttt{NIU:} }\textit{1388471})
\end{tcolorbox}	
\section{Resumen}
En el presente documento se comentan las soluciones al ejercicio final propuestos en las sesiones de \textit{Seminarios de M�todos Num�ricos} de los d�as 6, 13  y 20 de Mayo. Los ficheros \texttt{.c} con las implementaciones del c�digo en lenguaje C se adjuntan tal como se detalla en documento del mismo subdirectorio \texttt{README.txt}.\footnote{En el apartado, no numerado, Anexo se responden a algunos de los ejercicios complementarios.}
\footnotesize
%\tableofcontents
%\setcounter{section}{+3}
\normalsize
\section{Ejercicio}
 Tal como se detalla en el primer punto del \textit{procedimiento general}, en este caso mediante el m�todo de la Bisecci�n modificado, bajo las suposiciones del Lema 1 ["...."] obtenemos, dado un polinomio, los $n$ intervalos disjuntos dos a dos  en los que se encuentran sus ra�ces.
\end{document}
